\part*{Task 3: Identify Competitor and Similar Apps and List Them in Your Report}
\addcontentsline{toc}{part}{Task 3: Identify Competitor and Similar Apps and List Them in Your Report}

\begin{table}[H]
    \centering
    \begin{tblr}{
        width = \linewidth,
        colspec = {Q[246]Q[158]Q[279]Q[254]},
        cell{2}{1} = {r=2}{},
        cell{4}{1} = {r=2}{},
        vlines,
        hline{1-2,4,6} = {-}{},
        hline{3,5} = {2-4}{},
    }
                             & \textbf{App Name}     & \textbf{Link to Google Play}                                                                                                                                                                           & \textbf{Short Description} \\
    \textbf{Similar Apps}    & Uniper                & \href{https://play.google.com/store/apps/details?id=com.uniper.uniapp}{com.uniper.uniapp}                                                                                                              & Personalized virtual senior living community.\\
                             & Facebook              & \href{https://play.google.com/store/apps/details?id=com.facebook.katana&hl=en_CA}{com.facebook.katana}                                                                                                 & Social media to connect with family, friends, and various groups and organizations.\\
    \textbf{Competitor Apps} & Claris for Caregivers & \href{https://play.google.com/store/apps/details?id=com.clariscompanion.caregiver}{com.clariscompanion} \href{https://play.google.com/store/apps/details?id=com.clariscompanion.caregiver}{.caregiver} & Companion app for Claris tablet, provides a private social network and regular check-ins with healthcare providers.\\
                             & WhatsApp Messenger    & \href{https://play.google.com/store/apps/details?id=com.whatsapp}{com.whatsapp}                                                                                                                        & General instant messaging and calling app for groups of people.
    \end{tblr}
\end{table}

When defining similar and competitor apps, both categories include apps that
    meet many of the functional requirements of GrandPad.
However, the deciding factor was whether the the app could be negatively
    impacted by GrandPad's success (is a negative stakeholder).
If so, then it is labelled a competitor app.

For an app to be a negative stakeholder, users of GrandPad must be able to
    transition to the other platform without losing much of their functionality.

In the case of Claris for Caregivers, the app provides a private social network
    and a special tablet for the elderly user, similar to GrandPad, with access
    to comprehensive health check-ins.
Users transitioning from GrandPad to Claris would still have most of their needs
    satisfied, including video and voice calls, private networks, simple user
    interfaces, and connections across geographical barriers.

A transition to WhatsApp Messenger would yield similar results.
Although users would no longer enjoy protection from scams and would have
    privacy concerns, users would still be able to make video and voice calls,
    share posts, and send messages through a simple interface with high quality
    and reliability.
Research conducted during the COVID-19 pandemic, where social isolation was the
    norm, showed that WhatsApp messenger received praise from elderly users for
    its ease of use and agility for conducting day-to-dat communication
    \cite{llorente}.

Uniper and Facebook do not share this ability.
Uniper, providing a virtual senior living community, lacks many of
    functionalities that GrandPad provides, like private social networks and
    sharing posts with family and friends.
Facebook, providing a more traditional social media platform, has too many
    features distracting from the purpose of staying connected with family and
    friends.
Research during the COVID-19 pandemic revealed mixed results when analyzing the
    loneliness-moderating effect of traditional social media apps like Facebook
    \cite{longjing}.
While either of these two apps provide similar services to GrandPad, users would
    not be able to switch to them without significant reduction in desired
    functionalities.

This method of choosing similar and competitor apps identifies apps that perform
    similar functionalities to GrandPad and can highlight areas that make
    GrandPad stand out or improve upon.
For example, Facebook, being identified as a similar app that is not targeted
    enough to compete with GrandPad, could be used as a comparison, where
    GrandPad's privacy, content safety, and lack of advertisements are
    emphasized.
A competitor app, like Claris for Caregivers, can be used to identify that
    GrandPad is missing comprehensive health features, such as connecting with
    healthcare providers.

This approach lacks in being primarily conceptual, however.
No usage statistics, rating, or topic modelling were used to confirm these
    categorizations.
For example, it may be discovered that Claris for Caregivers is not a competitor
    because it has a low user satisfaction rate.
Additionally, the reliance of this method on manual searching for similar and
    competitor apps is not comprehensive and likely resulted in missed
    opportunities in finding additional apps for these categories.
